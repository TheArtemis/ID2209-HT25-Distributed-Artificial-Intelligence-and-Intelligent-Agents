\documentclass[a4paper,12pt]{article}

% --- PACKAGES ---
\usepackage[utf8]{inputenc}
\usepackage{graphicx}
\usepackage{hyperref}
\usepackage{fancyhdr}
\usepackage{titlesec}
\usepackage{geometry}
\usepackage{xcolor}
\usepackage{enumitem}

% --- PAGE SETUP ---
\geometry{margin=2.5cm}
\setlength{\parskip}{0.7em}
\setlength{\parindent}{0pt}
\pagestyle{fancy}
\fancyhf{}
\rhead{DAIIA Assignment Report}
\lhead{Deflorian, Fragale, Skarbalius}
\cfoot{\thepage}

% --- SECTION FORMATTING ---
\titleformat{\section}{\large\bfseries\color{blue}}{\thesection.}{0.5em}{}
\titleformat{\subsection}{\bfseries}{\thesubsection}{0.5em}{}
\titleformat{\subsubsection}[runin]{\bfseries}{\thesubsubsection}{0.5em}{}[.]

% --- TITLE ---
\title{\textbf{DAIIA Assignment Report}\\[0.5em]
\large Course: [Distributed Artificial Intelligence and AI Agents] \\[0.3em]
Assignment: Homework 2}
\author{Lorenzo Deflorian, Riccardo Fragale, Juozas Skarbalius \\[0.3em]
KTH Royal Institute of Technology \\[0.3em]
}
\date{\today}
\begin{document}

\maketitle
\tableofcontents
\newpage


% --- SECTION 1 ---
\section{Running Instructions}
Import the ZIP file given on GAMA and go to the folder assignment2/models. You will find a file
called FestivalAuction.gaml where our model for the homework is implemented.
Then, from the interface of GAMA, click on the play button and you will see a simulation.
Use the tools provided by the GAMA simulation interface to adjust the speed, read the outputs
and verify on screen that everything is working correctly.
Note that changing the parameters \textit{numCenter,numShop and numGuests} will change the number of 
agents that will appear on the screen and be part of the simulation. These parameters are still related to 
the implementation of the first homework but since we are introducing auctions
in the previous model, they might be still relevant for simulation purposes.

% --- SECTION 2 ---
\section{General Overview}
\subsection{New Agents}
In addition to the previous model of the Festival we needed to add new agents that are repsonsible for 
running auctions on several different products. 
As far as the basic implementation and the first challenge we have an agent called \textbf{Auctioneer}
that is responsible for the so-called Dutch auctions.
It appears on the map in a random location and it is a blue square of dimension 5.
For the purpose of the second challenge we implemented two new agents, called \textbf{EnglishAuctioneer} and \textbf{VickreyAuctioneer} that
are responsible for English and Vickrey auctions. 
Since for challenge 2 we would need to compare the profitability of each type of auction for auctioneers and bidders (guests)
we decided that each auctioneer sell the same products as the others.
In particular, products on auctions are alcohol, sugar and ashtonishings (they in a bit recall the environment of a techno music festival).

\subsection{Assumption}
In our festival we would have only three auctioneers and, at least for the case of the Dutch Auctioneer,
it is also able to showcase multiple products at the same time.
As a general rule, all interactions between auctioneers and bidders will follow FIPA protocol, instead of pure ask commands.
The auctioneers will appear on display at the beginning of the simulation, but they will start
the auction procedure after a certain randomic time defined as \textit{rnd(15,1500)}.
Each product will be sold just once by each auctioneer.


\subsection{Goals}
The four main goals for this homework are the following:
\begin{itemize}
    \item More experience with agents in GAMA
    \item Introduction to message passing and FIPA protocol
    \item Experience working with agent negotiation
    \item Simulating and practicing in an auction
\end{itemize}


% --- SECTION 3 ---
\section{Basic implementation}



\subsection*{3.1 Explanation}
As said before in the general overview we were aske to implement a new agent, called \textbf{Auctioneer} that is responsible 
for holding Dutch auctions of certain products inside the festival.
Just as a brief remainder, a Dutch auction starts with an offer by the auctioneer (in our case a randomic value) with much higher price than the expected market value.
If no one wants to buy for the set price, the auctioneer reduces the price at selected interval.
He is completely free to decide how much the price is reduced in every round.
If the price gets reduced below a minimum threshold, the auction is cancelled and the product is not sold.

\subsection*{3.2 Code}
Include relevant code snippets below:

\begin{verbatim}
# Example code snippet
def example():
    print("Hello, DAIIA!")
\end{verbatim}



\subsection*{3.3 Demonstration}
Provide two use cases for this section.

\begin{enumerate}[leftmargin=*]
  \item \textbf{Input:} Describe the input.
  \item \textbf{Screenshot:} Include program execution/output.
  \item \textbf{Interpretation:} Briefly explain the result.
\end{enumerate}

% --- SECTION 4 ---
\section{Challenge 1: Multiple Auctions in the Festival}

\subsection*{4.1 Explanation}
In this first challenge we were asked to allow multiple auctions at the same time. In our case 
we decided to allow a multiple auctioneer to setup multiple auctions at the same time of different products.
We would also show that guests are interested into different products and so they would bid offers only for what 
they are really interested. This was also clear in our basic implementation 
but since it is required specifically for this challenge we are highlighting it again.

\subsection*{4.2 Code}
Show relevant code snippets and screenshots.

\subsection*{4.3 Demonstration}
Provide 4 complete use cases:
\begin{enumerate}[leftmargin=*]
  \item Input description.
  \item Screenshot of program execution/output.
  \item Short explanation of results.
\end{enumerate}


% --- SECTION 5---
\section{Challenge 2: Different auction settings}

\subsection*{4.1 Explanation}
For the purpose of this second challenge we were aske to implement at least two more type of auctions.
We decided to implement English auctions and Vickrey auctions.


\subsection*{4.2 Code}
Show relevant code snippets and screenshots.

\subsection*{4.3 Demonstration and comparison of prices}
Provide a couple of use cases and show that certain auctions might be more or less 
profitable for auctioneers and guests.

% --- SECTION 6 ---
\section{Final Remarks}
In conclusion, this assignment provided us a second oppurtunity to learn the basics of GAMA simulation modelling. 
Moreover, we were also to "dirty" our hands or harder tasks such as communicsating using FIPA protocol
and establishing auctions. This homework was definitely more interesting and challenging
and thus we are very happy of having been able to correctly deliver it.

\end{document}