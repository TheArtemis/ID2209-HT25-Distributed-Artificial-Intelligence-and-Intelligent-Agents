\documentclass[a4paper,12pt]{article}

% --- PACKAGES ---
\usepackage[utf8]{inputenc}
\usepackage{graphicx}
\usepackage{hyperref}
\usepackage{fancyhdr}
\usepackage{titlesec}
\usepackage{geometry}
\usepackage{xcolor}
\usepackage{enumitem}

% --- PAGE SETUP ---
\geometry{margin=2.5cm}
\setlength{\parskip}{0.7em}
\setlength{\parindent}{0pt}
\pagestyle{fancy}
\fancyhf{}
\rhead{DAIIA Assignment Report}
\lhead{Your Name}
\cfoot{\thepage}

% --- SECTION FORMATTING ---
\titleformat{\section}{\large\bfseries\color{blue}}{\thesection.}{0.5em}{}
\titleformat{\subsection}{\bfseries}{\thesubsection}{0.5em}{}
\titleformat{\subsubsection}[runin]{\bfseries}{\thesubsubsection}{0.5em}{}[.]

% --- TITLE ---
\title{\textbf{DAIIA Assignment Report}\\[0.5em]
\large Course: [Course Name] \\[0.3em]
Assignment: [Assignment Title]}
\author{[Lorenzo Deflorian...] \\[0.3em]
KTH Royal Institute of Technology \\[0.3em]
Email: \texttt{[ldef@kth.se ...]}}
\date{\today}

\begin{document}

\maketitle
\tableofcontents
\newpage

% --- SECTION 0 ---
\section{Important Notes (Read Before Starting)}
\begin{itemize}[leftmargin=*]
  \item At the end of each assignment description, you will find a list of deliverables.
  \item These deliverables define the key points that will be checked in your code and report.
  \item You must:
  \begin{itemize}
    \item Show in your source code that all deliverables are implemented.
    \item Demonstrate them in your use cases.
  \end{itemize}
  \item Missing deliverables will require resubmission of a complete report.
\end{itemize}

% --- SECTION 1 ---
\section{Running Instructions}
\begin{itemize}[leftmargin=*]
  \item \textbf{How to run the program:} Describe the exact steps.
  \item \textbf{Expected inputs:} Explain what inputs are needed and how to provide them.
  \item \textbf{Important:}
  \begin{itemize}
    \item Instructions must work exactly as written.
    \item You are responsible for uploading:
    \begin{itemize}
      \item This report (PDF only)
      \item Complete and working source code
      \item Any additional files needed to run the program
    \end{itemize}
  \end{itemize}
\end{itemize}

% --- SECTION 2 ---
\section{General Overview}
\subsection{Solution Summary}
Summarize your overall approach. Mention assumptions or limitations here.

% --- SECTION 3 ---
\section{Section Reports}
Repeat the structure below for each part of the assignment.

\subsection*{Section X: [Section Name]}
\subsubsection*{3.1 Explanation}
Describe what this section requires and how you approached solving it.

\subsubsection*{3.2 Code}
Include relevant code snippets below:

\begin{verbatim}
# Example code snippet
def example():
    print("Hello, DAIIA!")
\end{verbatim}

Optionally include screenshots with:
\begin{center}
  \includegraphics[width=0.8\textwidth]{example_code.png}
\end{center}

\subsubsection*{3.3 Demonstration}
Provide two use cases for this section.

\begin{enumerate}[leftmargin=*]
  \item \textbf{Input:} Describe the input.
  \item \textbf{Screenshot:} Include program execution/output.
  \item \textbf{Interpretation:} Briefly explain the result.
\end{enumerate}

% --- SECTION 4 ---
\section{Challenge / Bonus Section (Optional)}
If you attempted bonus tasks, describe them here.

\subsection*{Challenge X: [Challenge Name]}
\subsubsection*{4.1 Explanation}
Explain the challenge and your approach.

\subsubsection*{4.2 Code}
Show relevant code snippets and screenshots.

\subsubsection*{4.3 Demonstration}
Provide 4 complete use cases:
\begin{enumerate}[leftmargin=*]
  \item Input description.
  \item Screenshot of program execution/output.
  \item Short explanation of results.
\end{enumerate}

% --- SECTION 5 ---
\section{Final Remarks}
Summarize what you learned, any limitations, and potential improvements.
Optionally include any extra comments.

\end{document}
