\documentclass[a4paper,12pt]{article}

% --- PACKAGES ---
\usepackage[utf8]{inputenc}
\usepackage{graphicx}
\usepackage{hyperref}
\usepackage{fancyhdr}
\usepackage{titlesec}
\usepackage{geometry}
\usepackage{xcolor}
\usepackage{enumitem}

% --- PAGE SETUP ---
\geometry{margin=2.5cm}
\setlength{\parskip}{0.7em}
\setlength{\parindent}{0pt}
\pagestyle{fancy}
\fancyhf{}
\rhead{DAIIA Assignment Report}
\lhead{Your Name}
\cfoot{\thepage}

% --- SECTION FORMATTING ---
\titleformat{\section}{\large\bfseries\color{blue}}{\thesection.}{0.5em}{}
\titleformat{\subsection}{\bfseries}{\thesubsection}{0.5em}{}
\titleformat{\subsubsection}[runin]{\bfseries}{\thesubsubsection}{0.5em}{}[.]

% --- TITLE ---
\title{\textbf{DAIIA Assignment Report}\\[0.5em]
\large Course: [Distributed Artificial Intelligence and AI Agents] \\[0.3em]
Assignment: [Homework 1]}
\author{[Lorenzo Deflorian, Riccardo Fragale, Jouzas Skarbalius] \\[0.3em]
KTH Royal Institute of Technology \\[0.3em]
}
\date{\today}

\begin{document}

\maketitle
\tableofcontents
\newpage

% --- SECTION 1 ---
\section{Running Instructions}
Import the ZIP file given on GAMA and go to the folder assignment1/models. You will find a file
called Festival.gaml where our model for the homework is implemented.
Then, from the interface of GAMA, click on the play button and you will see a simulation.
Use the tools provided by the GAMA simulation interface to adjust the speed, read the outputs
and verify on screen that everything is working correctly.
Note that changing the parameters \textit{numCenter,numShops and numGuests} will change the number of 
agents that will apply on the screen and be part of the simulation. 

% --- SECTION 2 ---
\section{General Overview}

\subsection{Agents}
We have defined 4 main species:
\begin{itemize}
    \item InformationCenter
    \item Shop 
    \item SecurityGuard
    \item Guest
\end{itemize}
We also have a subspecies of Guest that is called SmartGuest and has been implemented to obtain the objectives
requested by challenge 1.

The agent of type \textbf{InformationCenter} is responsible to give informations to all the agents that are asking him where the shops 
are. He knows the locations of them and is also connected to a SecurityGuard that will be helpful for the 
purpose of the second challenge. regarding the aspect of this agent, it is depicted as a black square of length 5 and its locations
is right in the center of the field where the simulation is happening. All the other agents know by default the position of the InformationCenter.

A \textbf{Shop} is an agent responsible of replenishing food or water to all the guests that are coming to them.
They are divided into food shops and water shops and their position changes in each simulation.
Water shops are depicted as grey triangles while food shops are red triangles.


The \textbf{Guest}, instead, takes part in the festival and basically wonders randomically unless it is 
either hungry or thirsty. In this case it goes to the InformationCenter and asks it
where is it possible to find food or water to fulfill its need. Then he starts moving to the location
given by the InformationCenter and "charges its batteries". Then, it continues moving aimlessly unless 
it feels hungry or thirsty again(or both). A guest is a pink dot.

As we said before, there is also the \textbf{SmartGuest} that has a memory and so it is able to remember the locations
of the shops visited. In this way he could move less with respect to normal guests and it will satisfy its need faster.
Said that, randomically it can go to the InformationCenter because he wants to discover new shops. The SmartGuest
appears on screen as a yellow dot.

Last but not the least, there is the \textbf{SecurityGuard} that is repsonsible to catch and kill all the guests
that doesn't follow the rules of the festival and must be eliminated.
The security guard is a blue dot and its starting position is northwest with respect to the InformationCenter.

\subsection{Assumptions}
Our model has 1 InformationCenter, 1 SecurityGuard, 4 Shops (two of which of type \textit{food} and two of type \textit{water}).
Regarding the Guests, we have defined three Smart guests and seven guests without memory.

\subsection{Goals}
The goals of this homework are:
\begin{itemize}
    \item Introduction to the Gama platform
    \item Working with agents
    \item Learning the GAMA syntax
    \item Creating different types of agents
    \item Starting basic simulations
    \item Little bit of movement and behaviour
\end{itemize}



% --- SECTION 3 ---
\section{Basic implementation}

\subsubsection*{3.1 Explanation}
The basic implementation required us to create a simulation of a festival where Guests get hungry or 
thirsty. If they do, they should go to an Information Center to ask for the 
nearest Store that gives them what they need. Afterward, the Guests 
should simply keep doing something until they get hungry/thirsty again (they wander randomically).

\subsubsection*{3.2 Code}
Include relevant code snippets below:

\begin{verbatim}
# Example code snippet
def example():
    print("Hello, DAIIA!")
\end{verbatim}

Optionally include screenshots with:


\subsubsection*{3.3 Demonstration}
Provide two use cases for this section.

\begin{enumerate}[leftmargin=*]
  \item \textbf{Input:} Describe the input.
  \item \textbf{Screenshot:} Include program execution/output.
  \item \textbf{Interpretation:} Briefly explain the result.
\end{enumerate}

% --- SECTION 4 ---
\section{Challenge / Bonus Section (Optional)}
If you attempted bonus tasks, describe them here.

\subsection*{Challenge X: [Challenge Name]}
\subsubsection*{4.1 Explanation}
Explain the challenge and your approach.

\subsubsection*{4.2 Code}
Show relevant code snippets and screenshots.

\subsubsection*{4.3 Demonstration}
Provide 4 complete use cases:
\begin{enumerate}[leftmargin=*]
  \item Input description.
  \item Screenshot of program execution/output.
  \item Short explanation of results.
\end{enumerate}

% --- SECTION 5 ---
\section{Final Remarks}
Overall this assignment was great. We had the opportunity to learn the basics of the syntax of Gamma
and we were challenged to implemented interesting things.
We realized that changing a bit the inputs or minor details might lead
to big changes in each simulation.


\end{document}